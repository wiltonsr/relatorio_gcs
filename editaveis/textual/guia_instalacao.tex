\chapter[Guia de Instalação]{Guia de Instalação}
  
  Esse capítulo apresenta um guia para a instalação do SVN mostrando como instalar, a documentação disponível e as
  principais dificuldades encontradas no processo de instalação.
    

\section{Instalação da ferramenta}
    
    O jeito mais fácil de instalar o Subversion em sistemas Unix-like, como o Ubuntu,
    é utilizando o sistema de pacotes nativo do sistema. Portanto, para a instalação do Subversion,
    utilize o gerenciador de pacotes do Ubuntu \textit{apt-get}, executando o seguinte comando:
    
    \colorbox{Gray}{
      \begin{minipage}{0.6\linewidth}
      \flushleft{ 
      \textbf{sudo apt-get install subversion libapache2-svn}
      }
      \end{minipage}
    }
    
    Com este comando, obtém-se também o pacote de configuração do servidor Apache para o uso com o Subversion.
  
\section{Disponibilidade de documentação de instalação}
  
  A Fundação Apache mantém uma vasta documentação para a instalação do Subversion de diferentes maneiras
  para diferentes sistemas operacionais \footnotemark, porém a documentação não é muito clara e o Subversion necessita
  \footnotetext{Disponível em: \url{http://svn.apache.org/repos/asf/subversion/trunk/INSTALL}.}
  de algumas bibliotecas auxiliares para o correto funcionamento \cite{svn-book}, o que pode complicar demais o processo
  de instalação manual utilizando o código fonte. Portanto, para a instalação do Subversion no Ubuntu, foi utilizado o
  tutorial fornecido em \citeonline{wiki-svn}.
  
  
\section{Principais dificuldades encontradas}
  
  A maior dificuldade encontrada foi na tentativa de instalar o Subversion utilizando o código fonte, pois dependia de 
  várias bibliotecas auxiliares e, mesmo depois de obter essas bibliotecas, não foi possível concluir a instalação da
  ferramenta.