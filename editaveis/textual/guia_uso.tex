\chapter[Guia de Uso]{Guia de Uso}

\section{Passo a passo para o primeiro contato com a ferramenta}

Agora que o SVN já está instalado, para criação de um repositório deve ser dado o seguinte comando:
"svnadmin create <camminho do repositorio>"


Os repositórios são identificados através de URLs.

A ferramenta trabalha com Cópias de trabalho. Uma cópia de trabalho consiste
na mesma estrutura do repositório, porém é utilizado para uso privado. Ou seja, é
a cópia local do repositório que pode ser editada pelo usuário.

Para que os outros membros do projeto tenham acesso ao que foi produzido na cópia
de trabalho, o SVN permite que o usuário escreva no repositório as alterações feitas
e para ter acesso às atualizações feitas por outros membros, é permitido ao usuário ler o 
repositório.

A pasta .svn é criada em cada cópia de trabalho, pois é ela que permite ao SVN reconhecer que há mudanças realizadas nos arquivos que precisam ser submetidas ao repositório.

Para escrever as alterações no repositório é realizado um:
"svn commit <nome do arquivo modificado> -m <"mensagem\">"

Para ler as alterações é realizado um:
"svn update"

Para mudanças no repositório e na cópia de trabalho os comandos devem ser realizados na seguinte ordem:
* svn update
* svn commit

Para ver os arquivos modificados o comando é: "svn status"

Para mostrar as diferenças m um arquivo "svn diff"

Para criar uma cópia de trabalho o comando é "svn checkout"

Para ver as revisões criadas o comando é: "svn status --verbose"

Para mostrar o histórico de mudanças o comando é "svn log"

Para desfazer as alterações em um arquivo: "svn revert <arquivo>"

Para criar uma branch, deve ser ser feita uma cópia da pasta trunk para a pasta de branchs com o seguinte comando:

"svn copy <caminho da pasta original> \ <caminho da pasta de destino, essa pasta deve possuir o nome da branch> -m "mensagem\">"


% A Subversion client commits (that is, communicates the changes made to) any number of files and directories as a single atomic
% transaction. By atomic transaction, we mean simply this: either all of the changes are accepted into the repository, or none of them
% is. Subversion tries to retain this atomicity in the face of program crashes, system crashes, network problems, and other users' actions.
% Each time the repository accepts a commit, this creates a new state of the filesystem tree, called a revision. Each revision is assigned
% a unique natural number, one greater than the number assigned to the previous revision. The initial revision of a freshly created
% repository is numbered 0 and consists of nothing but an empty root directory.
% Figure 1.6, “Tree changes over time” illustrates a nice way to visualize the repository. Imagine an array of revision numbers, starting
% at 0, stretching from left to right. Each revision number has a filesystem tree hanging below it, and each tree is a “snapshot” of
% the way the repository looked after a commit.
% 
% 
% Subversion client programs use URLs to identify versioned files and directories in Subversion repositories. For the most part, these
% URLs use the standard syntax, allowing for server names and port numbers to be specified as part of the URL.
% 
% A Subversion working copy is an ordinary directory tree on your local system, containing a collection of files. You can edit these
% Fundamental Concepts
% 9
% files however you wish, and if they're source code files, you can compile your program from them in the usual way. Your working
% copy is your own private work area: Subversion will never incorporate other people's changes, nor make your own changes available
% to others, until you explicitly tell it to do so. You can even have multiple working copies of the same project.
\section{Passo a passo para o uso das principais funcionalidades da ferramenta}

\section{Verificação das funcionalidades descritas a seguir em forma de pergunta:}

\begin{itemize}
  \item A ferramenta provê mecanismo de backup dos itens controlados pela ferramenta?

    Sim. O SVN provê um comando que possibilita a realização de um backup do repositório, caso algo dê errado. O comando é:
    "svnadmin hotcopy <caminho do repositorio> <caminho de backup>"

  \item A ferramenta provê algum mecanismo de controle de acesso?
    Sim. No arquivo "svnserve.conf" são descritos os níveis de acesso para usuários autenticados e não autenticados

  \item A ferramenta controla arquivos de diversos formatos, seja código, executáveis, documentação e diagramas?
    Sim. O SVN controla versão de todos os documentos colocados no repositório.

  \item A ferramenta registra versões dos itens de configuração?
    Sim. Cada alteração feita é registrada em um número de revisão. É possível visualizar um histórico de alterações com as seguintes informações:

    \subitem Nome do autor da alteração
    \subitem Data e hora
    \subitem Mensagem que descreve a alteração

  \item É possível visualizar as diferenças entre as versões incluindo a razão para estas 
  diferenças?

      O SVN permite a visualização das diferenças entre as versões, mas não apresenta a razão das diferenças.
  
  \item É possível identificar dependências entre artefatos, são elas: artefatos que pertencem a um mesmo item de configuração; itens de configuração de um componente; versão de itens de configuração de uma baseline; artefatos que são utilizados para construir um outro artefato  (por exemplo, um código é usado para construir a funcionalidade de um outro código)?
  
  
  \item É possível selecionar itens de configuração compatível com a versão válida e consistente do
  produto?
  \item É possível fazer snapshots ou congelar o estado de um produto a qualquer momento?
  \item A ferramenta provê mecanismos que facilitem a análise do impacto de se fazer uma mudança?
  \item É possível recuperar o histórico de todas as mudanças realizadas no itens controlados pela
  ferramenta?
  \item É possível recuperar o log de todos os detalhes do trabalho realizado?
  \item A ferramenta provê mecanismos para registrar estatísticas?
  \item A ferramenta provê mecanismos para examinar estado dos itens?
  \item A ferramenta provê mecanismos para selecionar aspectos para os quais deseja-se gerar relatório
  de acompanhamento de estado de configuração e gerar tal relatório?
  \item A ferramenta adverte sobre acesso inadequado a qualquer item para evitar mudanças não justificadas
  ou conflitos de mudanças?
  \item A ferramenta provê meios para monitorar bugs (quem, como e quando o bug foi gerado)?
  \item A ferramenta provê meios que facilitem a propagação de mudanças de maneira controlada
  através de diferentes versões de itens?
  \item A ferramenta provê mecanismos para facilitar a comunicação entre os interessados nos itens de
  configuração controlados?
  \item A ferramenta provê mecanismos para resolução de conflitos quando for necessário fazer merge
  de mudanças?
\end{itemize}
