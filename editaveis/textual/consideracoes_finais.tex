\chapter[Considerações Finais]{Considerações Finais}

\section{Resumo de cobertura das atividades relacionadas às funções da Gerência de Configuração}
 \subsection{Identificação da Configuração}
 \begin{itemize}
 \item \textbf{Requisitos de identificação de dados}
 \item \textbf{Identificar itens de configuação}
 \item \textbf{Estabelecer esquema de identificação}
 \item \textbf{Identificar requisitos de aceitação}
 \item \textbf{Definição das baselines}
 \end{itemize}
 \subsection{Controle de Mudanças}
 \begin{itemize}
 \item \textbf{Estabelecer revisões de controle}
 \item \textbf{EStabelecer procedimento de controle de mudança}
 \item \textbf{Controlar revisões}
 \end{itemize}
 \subsection{Auditoria}
 \begin{itemize}
 \item \textbf{Auditoria física}
 \item \textbf{Auditoria funcional}
 \item \textbf{Revisão formal}
 \end{itemize}
 \subsection{Relato de Estado}
 \begin{itemize}
 \item \textbf{Manter registro da definição dos produtos}
 \item \textbf{Manter registro da verificação da configuração}
 \item \textbf{Manter registro do estado das mudanças}
 \item \textbf{Manter histórico das aprovações das mudanças}
 \end{itemize}
\section{Resumo dos principais achados relacionados à instalação e uso da ferramenta}

	O processo de instalação utilizado para o Ubuntu é bastante simples, embora o processo de instalação manual por meio do 
	código fonte seja bem mais custoso, por depender de outras bibliotecas auxiliares que as vezes não estão inclusas.
	A ferramenta possui uma boa documentação tanto quanto para a instalação, quanto para a descrição das funcionalidades.
	O uso da ferramenta não é tão fácil no primeiro contato, principalmente para quem já está acostumado com outra ferramenta 
	de controle de versão como o Git, pois embora algumas funcionalidades sejam as mesmas, a estrutura de funcionamento difere muito.

	Destacamos abaixo os pontos positivos e negativos em relação à ferramenta.

	\subsection{Pontos positivos}

			Os pontos positivos identificados pelo grupo foram:

			\begin{itemize}
				\item Possui controle de acesso ao repositório;
				\item Gera um log com data e nome do autor que fez aquela alteração;
				\item Trabalha com a criação de \textit{branches};
			\end{itemize}


	\subsection{Pontos negativos}

			Os pontos negativos identificados pelo grupo foram:

			\begin{itemize}
				\item Necessidade do uso de um servidor próprio;
				\item Muitos passos para configurar;
				\item Não há possibilidade de realizar \textit{commits} locais;
			\end{itemize}

\section{O grupo usaria a ferramenta em seus projetos de desenvolvimento de software? Por quê?}
