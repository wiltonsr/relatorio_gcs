\chapter[Considerações Finais]{Considerações Finais}

Neste capítulo são apresentadas as considerações do grupo acerca da relação entre as características da ferramenta e a Gerência de Configuração, os
pontos positivos e negativos identificados e demais considerações sobre o uso e a instalação do SVN.

\section{Resumo de cobertura das atividades relacionadas às funções da Gerência de Configuração}

Nesta seção são estabelecidas as relações das atividades de cada uma das fases da Gerência de Configuração com a ferramenta.

 \subsection{Identificação da Configuração}
 \begin{itemize}
  \item \textbf{Identificar itens de configuração}

    A ferramenta permite que o usuário informe quais itens estão sob controle de versão.

  \item \textbf{Estabelecer esquema de identificação}

  \citeonline{svn-book} propõe uma estrutura de pastas recomendada (\textit{/branches}, \textit{/tags} e \textit{/trunk}).

  \item \textbf{Definição das \textit{baselines}}

  A ferramenta provê a possibilidade de estabelecer \textit{baselines} a partir da cópia dos arquivos do diretório \textit{/trunk},
  que não serão mais modificados enquanto eles estiverem sob \textit{baseline}, para o diretório \textit{/branches}. Para assim permitir a correção, atualização e evolução dos
  arquivos que continuam no diretório \textit{/trunk}.
 \end{itemize}
 \subsection{Controle de Mudanças}
 \begin{itemize}
  \item \textbf{Estabelecer revisões de controle e Controlar revisões}

     A ferramenta registra as mudanças por meio dos \textit{commits}.

  \item \textbf{Estabelecer procedimento de controle de mudança}

  A ferramenta permite que seja estabelecida um procedimento de controle de mudança por meio de uma política \textit{/branches}
  que pode ser definida a critério da equipe.

 \end{itemize}
 \subsection{Auditoria}

  Se tratando da utilização da ferramenta, não é possível realizar auditorias.

 \subsection{Relato de Estado}
 \begin{itemize}
  \item \textbf{Manter histórico das mudanças}

  O registro das mudanças são mantidos através dos \textit{commits} que podem ser recuperados através dos \textit{logs}.

 \end{itemize}
\section{Resumo dos principais achados relacionados à instalação e uso da ferramenta}

	O processo de instalação utilizado para o Ubuntu é bastante simples, embora o processo de instalação manual por meio do
	código fonte seja bem mais custoso, por depender de outras bibliotecas auxiliares que as vezes não estão inclusas.
	A ferramenta possui uma boa documentação tanto para a instalação, quanto para a descrição das funcionalidades.
	O uso da ferramenta não é tão fácil no primeiro contato, principalmente para quem já está acostumado com outra ferramenta
	de controle de versão como o Git, pois, embora algumas funcionalidades sejam as mesmas, a estrutura de funcionamento difere muito.

	\subsection{Pontos positivos}

			Os pontos positivos identificados pelo grupo foram:

			\begin{itemize}
				\item Possui controle de acesso ao repositório;
				\item Gera um \textit{log} com data e nome do autor que fez aquela alteração;
				\item Trabalha com a criação de \textit{branches};
				\item É \textit{open source};
				\item Possui uma boa documentação;
			\end{itemize}


	\subsection{Pontos negativos}

			Os pontos negativos identificados pelo grupo foram:

			\begin{itemize}
				\item Necessidade do uso de um servidor próprio;
				\item Muitos passos para configurar;
				\item Não há possibilidade de realizar \textit{commits} locais;
			\end{itemize}

\section{O grupo usaria a ferramenta em seus projetos de desenvolvimento de software? Por quê?}

 Nenhum integrante da equipe do trabalho apoiaria a utilização da ferramenta em um projeto atual, pois atualmente existem tecnologias que
 oferecem melhores recursos para a gerência de configuração, no caso de ferramentas \textit{open source}, como o Git. Ou Ainda soluções comerciais, como o ClearCase da IBM.
